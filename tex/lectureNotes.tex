%{{{ Preamble
\nonstopmode
\documentclass{beamer}
%\usepackage{xgreek}
\usepackage{xunicode}
\usepackage{hyperref}
\usepackage{float}
\usepackage{xltxtra}
\usepackage{setspace}
\usepackage{graphicx}
\usepackage{caption}
\usepackage{subcaption}
\usepackage{minted}
\usepackage{amssymb}
\usepackage{txfonts}
\usepackage{tabularx}
\usepackage{mathtools}
%\setsansfont[Mapping=TeX-text]{CMU Sans Serif}
\setsansfont[Mapping=TeX-text]{FreeSans}
\definecolor{officegreen}{rgb}{0.0, 0.5, 0.0}
%}}}

%{{{ Themes
\usetheme{umbc1} % good one
%\usetheme{umbc2} % good one
%\usetheme{default}
%\usetheme{PaloAlto}
%\usetheme{Warsaw} %<-- Well known blue theme
%\usetheme{Berlin}
%\usecolortheme{seahorse} % good one
%\usecolortheme{rose}
%\usetheme{Berkeley}
%\usetheme{boxes}
%\usetheme{Antibes} % good one
%\usetheme{Bergen}
%\usetheme{CambridgeUS}
%\usetheme{Copenhagen}
%\usetheme{Darmstadt}
%\usetheme{Dresden}
%\usetheme{Frankfurt}
%\usetheme{Goettingen}
%\usetheme{Hannover}
%\usetheme{Ilmenau}
%\usetheme{JuanLesPins}
%\usetheme{Luebeck}
%\usetheme{Madrid}
%\usetheme{Malmoe}
%\usetheme{Marburg}
%\usetheme{Montpellier} % simple enough, but looks ok
%\usetheme{Pittsburgh}
%\usetheme{Rochester}
%\usetheme{Singapore}
%\usetheme{Szeged}
%\usetheme{Boadilla}
%\usetheme{AnnArbor}
%}}}

\title[Προγραμματισμός Ηλεκτρονικών Υπολογιστών]{Προγραμματισμός Ηλεκτρονικών Υπολογιστών}
\author[Σ.Ζάχος, Ν.Παπασπύρου]{Σ.Ζάχος, Ν.Παπασπύρου}
%\institution[NTUA]{National Technical University of Athens}
\date[12/13]{6/12/13}

\begin{document}

\begin{frame}
	\titlepage
\end{frame}

\begin{frame}
  \frametitle{Περιεχόμενα}
    \tableofcontents%[pausesections]
\end{frame}

%{{{ Περιγραφή μαθήματος
\begin{frame}
	\frametitle{Προγραμματισμός Ηλεκτρονικών Υπολογιστών\\
		\url{http://courses.softlab.ntua.gr/progintro/}}

	Διδάσκοντες:
	\begin{tabular}{ll}
	  Στάθης Ζάχος & (\href{mailto:zachos@cs.ntua.gr}{\nolinkurl{zachos@cs.ntua.gr}}) \\
	  Νίκος Παπασπύρου & (\href{mailto:nickie@softlab.ntua.gr}{\nolinkurl{nickie@softlab.ntua.gr}}) \\
	  Δημήτρης Φωτάκης & (\href{mailto:fotakis@cs.ntua.gr}{\nolinkurl{fotakis@cs.ntua.gr}})
	\end{tabular}
	\vskip20pt

	\begin{center}{\large Διαφάνειες Παρουσιάσεων}\end{center}
	\begin{itemize}
	  \item[\checkmark] Εισαγωγή στην πληροφορική
		\item[\checkmark] Εισαγωγή στον προγραμματισμό με τη γλώσσα
			\textbf{{\color{officegreen}Pa{\color{red}z}cal}}
	  \item[\checkmark] Μεθοδολογία αλγοριθμικής επίλυσης προβλημάτων
	\end{itemize}
\end{frame}
%}}}

%{{{ Εισαγωγή
\section{Εισαγωγή}
\begin{frame}
	\frametitle{Εισαγωγή}
	\only<1>{
		\framesubtitle{(i)}
		\begin{itemize}
			\item[$\Diamondblack$]{\large Σκοπός του μαθήματος}
			\begin{itemize}
				\item Εισαγωγή στην {\color{violet}πληροφορική} (computer science)
				\item Εισαγωγή στον {\color{violet}προγραμματισμό} ηλεκτρονικών
					υπολογιστών (Η/Υ)
				\item Μεθοδολογία {\color{violet}αλγοριθμικής επίλυσης προβλημάτων}
			\end{itemize}
		\end{itemize}
	}

	\only<2>{
		\framesubtitle{(ii)}
		\begin{itemize}
			\item[$\Diamondblack$]{\large Αλγόριθμος}
			\begin{itemize}
				\item Πεπερασμένη ακολουθία {\color{violet}ενεργειών} που περιγράφει τον
					τρόπο επίλυσης ενός προβλήματος
				\item Εφαρμόζεται σε {\color{violet}δεδομένα} (data)
			\end{itemize}
			\item[$\Diamondblack$]{\large Πρόγραμμα}
			\begin{itemize}
				\item Ακριβής περιγραφή ενός αλγορίθμου σε μία {\color{violet}τυπική γλώσσα}
					που ονομάζεται {\color{violet}γλώσσα προγραμματισμού}
			\end{itemize}
		\end{itemize}
	}

	\only<3>{
		\framesubtitle{(iii)}
		\begin{itemize}
			\item[$\Diamondblack$]{\large Φυσική γλώσσα}
			\begin{itemize}
				\item Χωρίς τόσο αυστηρούς {\color{violet}συντακτικούς} περιορισμούς
				\item Μεγάλη πυκνότητα και {\color{violet}σημασιολογική} ικανότητα
			\end{itemize}
			\item[$\Diamondblack$]{\large Τυπική γλώσσα}
			\begin{itemize}
				\item {\color{violet}Αυστηρότατη} σύνταξη και σημασιολογία
			\end{itemize}
			\item[$\Diamondblack$]{\large Γλώσσα προγραμματισμού}
			\begin{itemize}
				\item Τυπική γλώσσα στην οποία μπορούν να περιγραφούν
					{\color{violet}υπολογισμοί}
				\item {\color{violet}Εκτελέσιμη} από ένα ηλεκτρονικό υπολογιστή
			\end{itemize}
		\end{itemize}
	}

	\only<4>{
		\framesubtitle{(iv)}
		\begin{itemize}
			\item[$\Diamondblack$]{\large Πληροφορική}
				\noindent\hspace*{-0.5cm}\begin{tabularx}{\textwidth+1cm}{
						>{\advance\hsize-4em}X
						>{\advance\hsize-2em}X
						>{\advance\hsize1em}X
				}
					{\color{violet}Ηλεκτρονικοί \newline Υπολογιστές \newline (engineering)} &
					$\newline\xleftrightarrow{\mathmakebox[8em]{ }}$ & $\newline${\color{violet}Μαθηματικά} \\
					 & & \\
					Σχεδίαση και \newline κατασκευή & & Θεωρία και \newline αναλυτική μέθοδος
				\end{tabularx}
				\vskip20pt
			\item[$\Diamondblack$]{\large Κεντρική έννοια:\\
					\hskip20pt {\color{violet}υπολογισμός} (computation)}
		\end{itemize}
	}
\end{frame}

%}}}

%{{{ Γλώσσες Προγραμματισμού
\section{Γλώσσες Προγραμματισμού}
\begin{frame}
	\frametitle{Γλώσσες Προγραμματισμού}
  To be done
\end{frame}
%}}}

%{{{ Ασκήσεις
\section{Ασκήσεις}

\subsection{Ασκήσεις (Pascal)}
\begin{frame}
	\frametitle{Ασκήσεις (Pascal)}
  To be done
\end{frame}

\subsection{Ασκήσεις (C)}
\begin{frame}
	\frametitle{Ασκήσεις (C)}
  To be done
\end{frame}
%}}}

%{{{ Δομή του προγράμματος
\section{Δομή του προγράμματος}
\begin{frame}
  To be done
\end{frame}
%}}}

%{{{ Τι σημαίνει ορθό πρόγραμμα
\section{Τι σημαίνει ορθό πρόγραμμα}
\begin{frame}
  To be done
\end{frame}
%}}}

\end{document}
